\chapwithtoc{Introduction}

% What is the nature of the problem the thesis is addressing?
Modern software development with rapid development cycles is in dire need of a fast, accessible, scalable, and reliable performance regression detection method.
A typical question is if neighboring commits of the same software manifest regression.
For instance, authors of a compiler might be concerned if a new optimization increased the performance of a particular benchmark.

% What is the common approach for solving that problem now?
The traditional method of performance measurement relies on a dedicated bare-metal server with an isolated and carefully set up environment that tries to prevent measurement noise.
This method is reasonably fast and reliable but lacks accessibility and scalability requirements as the cost of maintenance is high.

To address the scalability and accessibility much research over the past decade has focused on offloading performance measurements to the cloud~\cite{leitner2016patterns, laaber2019software, abedi2017conducting}.
Cloud offers scalable \mbox{on-demand} infrastructure with pay for what you use principle.
However, this research has shown that performance measurement in the cloud is difficult because of the high variability of the shared infrastructure environment.

Even in ideally crafted \mbox{bare-metal} conditions, multiple measurements are necessary to deal with the inherent variability introduced by the code running platform, such as \mbox{just-in-time} compilation, garbage collection, memory mapping, and CPU frequency scaling.
Benchmark code itself can have its share of non-determinism that contributes to the overall variability of the measurements.
Hence, in addition to benchmark harness running its workload in a loop practitioners often run the benchmark multiple times.
Afterward, analyze individual iteration scores to evaluate the final score.

In the context of compiler programmers, they may run multiple benchmarks on a version with and without the optimization and compare the final scores.
We refer to this method as \emph{sequential benchmarking} as it runs different versions after each other.

\citet{bulej2020duet} introduced a new benchmarking method called \emph{duet benchmarking} designed with volatile environments such as the cloud in mind.
The main idea of duet benchmarking is to run both versions in parallel.
The distinguishing property of this method is that it does not try to prevent interference, on the contrary, duet benchmarking acknowledges the presence of variability and relies on fair scheduling to make the impact equal on both versions running in parallel.
Afterward, the benchmark scores are statistically evaluated to determine if there is a performance regression between the two versions.
The duet benchmarking method has shown improvement ranging from $5.03 \times$ on average for Scalabench and DaCapo workloads to $37.4 \times$ on average for SPEC CPU 2017 workloads~\cite{bulej2020duet}.
However, this method relies on synchronized iterations of a benchmark, which requires benchmark harness modification that synchronizes individual iterations between harnesses.

% How this thesis approaches the problem?
This thesis explores the relaxation of this synchronization requirement which creates a new variant of the duet method we call \emph{asynchronous duet method}.
To achieve this, we implement a tool for running benchmarks that supports sequential, synchronous, and asynchronous duet methods.
One novel attribute of this tool is that it runs benchmarks in containers to achieve portability and ease of setup.

Using the tool we run benchmarks from multiple benchmarking suites --- Renaissance, Scalabench, DaCapo, and SPEC CPU 2017.
The first three are \mbox{JVM-based} while the last one represents natively compiled benchmarks.
The goal of this thesis is to assess the viability of the asynchronous duet method for different benchmarks across different environments.
Our choice of environments included bare-metal measurements on a dedicated server, AWS EC2 \lstinline{t3.medium} instance as cloud representative, and self-hosted shared virtual machine infrastructure.

% What are the results? Did something improve?
Results show that asynchronous duet performs very similarly to synchronous duet in terms of A/A detection accuracy.
It is also able to achieve minimal detectable slowdowns (MDS)~\cite{laaber2019software} of $1\%$ for SPEC CPU benchmarks and $3-5\%$ for \mbox{JVM-based} benchmarks.
Somewhat unexpectedly, our cloud measurements did not manifest variance much bigger than the \mbox{bare-metal} environment while the shared virtual machine environment came out as the most volatile.

Another surprising result was the performance of the sequential method, which performed well when it comes to the overall variability of iteration scores with a coefficient of variability being lower for many benchmarks.
However, the sequential method did not perform well in the statistical tests we chose for regression detection.

That relates to arguably the biggest advantage of the duet methods which is its overall runtime reduction compared to sequential methods.
Some benchmarks were able to finish in half the time using duet methods compared to sequential methods which directly translates to $50\%$ cost reduction.
Although, the speedup was dependent on the environment.
The cloud environment even experienced some slowdowns, likely due to fewer available CPUs.

% What can the reader expect in the individual chapters of the thesis?
The thesis is structured as follows.
The first chapter provides some background on benchmarking in volatile environments.
The second chapter delves into duet benchmarking and poses research questions for this thesis with the goals summarized at the end of this chapter.
The third chapter describes the architecture of our benchmark automation tool capable of running benchmarks from multiple suites using both sequential and duet methods.
The fourth chapter summarizes the statistical methods used to evaluate regressions which are later used in the fifth chapter that presents the results.
Finally, we conclude the thesis and lay out the possibilities for future work in this area.
